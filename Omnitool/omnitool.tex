% Omnitool cheaty na matiku by Plasmoxy

\documentclass[10pt,a4paper]{article}

\usepackage{a4wide}

\usepackage[slovak]{babel}
\usepackage[utf8]{inputenc}
\usepackage[sfdefault]{ClearSans}

\usepackage{graphicx}
\usepackage{url}
\usepackage{hyperref}
\usepackage{fancyhdr}
\usepackage{multicol}
\usepackage{blindtext}
\usepackage{titling}
\usepackage[left=1.5cm, right=1.5cm, top=1.5cm, bottom=2cm]{geometry}

\begin{document}

	% Titel
	\begin{center}
		\Large{\textbf{Omnitool} $\cdot$ v0.1}\\
		(cheaty na matiku)\\
		\today\ $\cdot$ by Plasmoxy
	\end{center}

	\section{Matematická Analýza}
	
	\subsection{Limity}
	
	\section{Algebra a diskrétna matematika}
	\subsection{Gausova el. metóda}
	\textbf{Elementárne riadkové operácie:}
	\begin{itemize}
		\item \textbf{ERO 1} - Výmena poradia 2 riadkov,
		\item \textbf{ERO 2} - Vynásobenie riadku nenulovou konštantou.
		\item \textbf{ERO 3} - Pripočítanie nenulového násobku jedného riatku k inému riadku.
	\end{itemize}
	
	

\end{document}